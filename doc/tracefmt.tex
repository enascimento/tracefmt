% =============================================================================

a trace is a sequence of samples

a container can be used to store one or more traces, representing either an
entire or partial data set (i.e., a given container might store a subset of 
traces in the data set)

the combination of all samples within all traces is termed the data, which
can be thought of as a sequence.  if there are n samples and t traces, st. 
s_{i,j} is the j-th sample from the i-th trace, a container organises the 
data in one of two ways: reading left-to-right and top-to-bottom these are

- trace-major 
  \[
  s_{  0,  0}, s_{  0,  1}, \ldots, s_{  0,n-1}
  s_{  1,  0}, s_{  1,  1}, \ldots, s_{  1,n-1}
  \vdots                    \ddots  \vdots
  s_{t-1,  0}, s_{t-1,  1}, \ldots, s_{t-1,n-1}
  \]
- sample-major
  \[
  s_{  0,  0}, s_{  1,  0}, \ldots, s_{t-1,  0}
  s_{  0,  1}, s_{  1,  1}, \ldots, s_{t-1,  1}
  \vdots                    \ddots  \vdots
  s_{  0,n-1}, s_{  1,n-1}, \ldots, s_{t-1,n-1}
  \]

the samples can be dense or sparse, st. each s_{i,j} is either

- a sample value (wrt. y-axis) whose offset along the x-axis is implicit 
  (i.e., a function of $j$ and an axis definition),
- a sample including both a value {\em and} an {\em explicit} offset along 
  the x-axis (i.e., irrespective of $j$).

a container can (optionally) be identified (e.g., with a timestamp), and
include a checksum (wrt. to the sample data itself) 

% =============================================================================

use-cases

- power analysis traces

- execution timings (e.g., for a cache-based attack)

- network traffic capture

% =============================================================================
