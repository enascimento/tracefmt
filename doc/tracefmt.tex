\documentclass[10pt,a4paper]{llncs}

\title{{\tt tracefmt}: a standard container format for side-channel data sets}
\author{J. Longo Galea \and L. Mather \and D. Page}
\institute{
Department of Computer Science, University of Bristol,\\
Merchant Venturers Building, Woodland Road,\\
Bristol, BS8 1UB, United Kingdom.\\
\email{\{Jake.Longo,Luke.Mather,Daniel.Page\}@bristol.ac.uk}
}

\begin{document}

\maketitle

% =============================================================================

In line with research funding bodies elsewhere, the UK-based EPSRC has clear 
policies\footnote{
\url{http://www.epsrc.ac.uk/about/standards/researchdata/}
} regarding research data, e.g., that ``publicly funded research data should 
generally be made as widely and freely available as possible''.  Especially
in topics relating to information security, where openness is arguably even 
more important, such policies make perfect sense.  However, without suitable
infrastructure, realising them can be problematic.  {\tt tracefmt} aims to 
address this problem within the field of (experimental) side-channel attack
by providing a standard, platform and application independent container 
format for on-disk storage of side-channel data sets.

To maximise uptake, the underlying design ethos is to favour portability and
ease of use over efficiency.  With this in mind, the specification makes use
of the {\sf ProtocolBuffers}\footnote{
\url{https://developers.google.com/protocol-buffers/}
} system.  On one hand, using {\sf ProtocolBuffers} means the {\tt tracefmt} 
specification is easy to understand and maintain, plus it allows automatic 
generation of a consistent implementation for {\em any} programming language.  
On the other hand, however, there {\e are} disadvantages wrt. efficiency: the
{\sf ProtocolBuffers} documentation even warns it is ``not designed to handle 
large messages'', and the tag-based binary format implies a significant 
storage overhead versus a more bespoke solution.  Most disadvantages do not
seem absolute, however: treating the {\tt tracefmt} specification as a 
specification only, it is of course possible to realise a specifically
 optimised non-{\sf ProtocolBuffers} implementation (e.g., to allow random 
rather than monolithic access to the data).  Also note that the focus is 
on-disk storage, not {\em necessarily} in-memory data structure.  While a
{\sf ProtocolBuffers}-based implementation will parse input to form a data
structure, this might be unsuitable for computationally intensive processing
and hence require a second translation step; we deem this a reasonable caveat
given the overarching goal.

\paragraph{Note.} 
This document and the associated specification are both {\em very} much in
development, rather than something you can use immediately; we welcome any
form of comment, suggestion or contribution.

% =============================================================================

A trace 
\[
T = < T_0, T_1, \ldots, T_{l-1} >
\]
is an $l$-element sequence of samples, with $T_i$ denoting the $i$-th such
element

the samples can be dense or sparse, st. each s_{i,j} is either

- a sample value (wrt. y-axis) whose offset along the x-axis is implicit 
  (i.e., a function of $j$ and an axis definition),
- a sample including both a value {\em and} an {\em explicit} offset along 
  the x-axis (i.e., irrespective of $j$).

A container can be used to store one or more traces, representing either an
entire or partial data set (i.e., a given container might store a subset of 
traces in the data set).  A container storing $n$ traces, where $T_{i,j}$ 
denotes the $i$-th sample from the $j$-th trace, organises them in one of 
two ways: reading left-to-right and top-to-bottom these are

\begin{enumerate}
\item trace-major 
      \[
      T_{  0,  0}, T_{  1,  0}, \ldots, T_{l-1,  0}
      T_{  0,  1}, T_{  1,  1}, \ldots, T_{l-1,  1}
      \vdots                    \ddots  \vdots
      T_{  0,n-1}, T_{  1,n-1}, \ldots, T_{l-1,n-1}
      \]
\item sample-major
      \[
      T_{  0,  0}, T_{  0,  1}, \ldots, T_{  0,n-1}
      T_{  0,  0}, T_{  0,  1}, \ldots, T_{  1,n-1}
      \vdots                    \ddots  \vdots
      T_{l-1,  0}, T_{l-1,  1}, \ldots, T_{l-1,n-1}
      \]
\end{enumerate}

\item A container can (optionally) include user-defined meta-data, specified
      via a database of key-value tuples; an additional index can assoicate 
      each key-value with one of the traces.  

      The unstructured nature of the meta-data forms a compromise: 

\item A container can (optionally) be identified (e.g., using a timestamp),  
      and include a checksum (on the sample data) .

% =============================================================================

use-cases

- power analysis traces

- execution timings (e.g., for a cache-based attack)

- network traffic capture

% =============================================================================

\end{document}
